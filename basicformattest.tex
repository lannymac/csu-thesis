\documentclass[draft]{csuthesis} %draft means that it won't actually render any pictures (graphs) it will just include space for them.
\title[Improved Multifractal Calculations]{Improved Methods for Calculating the Multifractal Spectrum for Small Data Sets}
\author{Leif Anderson}
\departmentname{Department of Physics}
\gradterm{Spring 2014} %The semester that you are going to turn this in.
\advisor{Richard Eykholt}
\coadvisor{Test Test} %optional
\committee{Mingzong Wu \and Raymond Steve Robinson \and Martin Gelfand \and Patrick Shipman} %separate committee names with \and
%committee names should not have any honorifics (i.e. NO Dr., PhD, professor, etc.)  Just names.
\copyrighttext{Copyright by Leif Anderson 2014 \\ All Rights Reserved} %this is optional.

%you do not need these newcommand lines.
\newcommand{\deriv}[2]{\frac{\mathrm{d}#1}{\mathrm{d}#2}}
\newcommand{\dby}[1]{\frac{\mathrm{d}}{\mathrm{d}#1}}
\newcommand{\pd}[2]{\frac{\partial#1}{\partial#2}}
\newcommand{\pby}[1]{\frac{\partial}{\partial#1}}

\usepackage{lipsum}%this package provides nonsense text for testing document layouts.  Not needed for real thesis.

\begin{document}
\frontmatter
%turns the page numbering to roman, and does some other stuff.

% \input{abstract}
\begin{abstract}
This is the abstract right here. I wanted a short input, so I typed these couple sentences, then I included some latin filler: \lipsum[1-4]

End of Abstract.
\end{abstract}
%note that for amsbook (which is basically the class we are copying), the abstract must be declared before the title.  It prints as part of maketitle.

\begin{acknowledgements}
Thanks to Dan Brake, Fran Campana, and Natalie Anderson for their help defeating the rules of the graduate school; to Katherine Zaunbrecher for her proofreading and motivation help; and to many others.
\end{acknowledgements}%this is optional.  It must be declared before maketitle, a lot like abstract.  Consider having this in a separate file.

\maketitle
% maketitle is a huge command in a small package.  It will render the title page, copyright page, acknowledgements, abstract, and probably a bunch of other stuff that I'm forgetting about.  If you choose not to define copyright, acknowledgements, etc, it automatically doesn't render them.

% other optional frontmatter could go somewhere in here.  dedication, biography, etc.  Grad school lists approved frontmatter pages somewhere.
% most frontmatter than I consider silly is not yet automated.  Also most frontmatter doesn't have any explicit style requirements, so I feel justified in ignoring it.

\tableofcontents %don't move this around too much
\listoftables %optional, but this is the spot it should go
\listoffigures %optional, but this is the right location for it

\mainmatter %switches page numbering to normal, resets page counter, etc.
\chapter{Introduction}

% \chapter{Introduction}
\section{Motivation}

One of the most interesting aspects of chaos is the universality of many of the results.  One of the most famous examples is Fiebenbaum's proof of the quantitative universality of his scaling factors for all quadratic maps.  This result proves that measurements of $\alpha$ and $\delta$ from the bifurcation diagrams of vastly different physical systems should converge to the same universal values.  The beauty of that result is that despite the very different manifestations, one can show that in a way, most chaotic systems are just expressions of a more unified type of behavior, and still bear some of the quantifiable characteristics of that behavior.

In a similar vein, it can be shown that %reference here
the fractal dimension of a chaotic attractor is invariant under smooth coordinate changes %check the phrasing and conditions.  fixed-mass paper, first ref.
Thus, one can measure a quantity based on the measurements corresponding to a particular set of initial conditions, with some very specific embedding method, and still reach a result that characterizes the underlying dynamics in a way that is completely universal.

% Another universal quantity is the fractal dimension of a chaotic attractor, which is invariant under smooth coordinate transformations. %need a ref here.  Check the fixed-mass paper intro, which is where I am stealing this whole idea from.  While we're at it, reference feigenbaum.
% There are multiple definitions of dimension, and they give conflicting results.  However, a wide class of definitions can be unified with the multifractal spectrum.  Calculations of the resulting spectrum of dimensions can require vast amounts of data. %ref something here?

One of the potential problems with this idea is that there are multiple definitions of dimension, and these definitions are known to differ from one another.  This can be resolved with the introduction of the multifractal spectrum, which provides a unifying framework for a very broad class of fractal dimensions.  However, accurate calculations of the multifractal spectrum usually require prohibitive amounts of data, especially for high-dimensional attractors.  Our aim is to improve calculations of the multifractal spectrum for small data sets.

\section{Problems with calculation of the Multifractal Spectrum}

As alluded to repeatedly, the multifractal spectrum is difficult to compute, requiring large amounts of data.  This slow convergence is especially pronounced for the negative $q$ portion of the spectrum.  This is because the negative $q$ part of the spectrum is controlled by the sparsest parts of the attractor, which by definition are the parts that are populated the slowest for most processes.

One interpretation of the multifractal spectrum $D_q$ is that the value of $q$ picks out which areas of the attractor to focus on.  This is an oversimplification, but it is phenominologically correct.  Recall that $D_q$ is found by examining $\Gamma=\sum\frac{p^q}{\ell^\tau}$.  In the $q\to+\infty$ limit, the terms that dominate the sum are the terms with the largest values of $p$; the densest parts of the attractor.  In that limit, $D_\infty$ can be loosely thought of as the dimension of the dense parts of the attractor.  Similarly, as $q\to-\infty$ limit, the terms that dominate the sum are the smallest values of $p$, which allows the interpretation of $D_{-\infty}$ as the dimension of the support of the sparsest areas of the attractor.  The more correct interpretation would be to examine $f(\alpha)$, and make arguments about the extremal values of $\alpha$, since $f$ really is defined as the dimension of the support of certain sets of points.  However, this looser interpretation makes a better link with the actual data.

Clearly, if the attractor is populated by a process that randomly places points selected according to the natural measure of the attractor %check phrasing
then these sparse areas will be rarely visited, making it hard to get an accurate estimate of the measure of the attractor contained in a box which only encompasses sparse parts of the attractor.  Further complicating matters is the fact that for any real measurement of a system, the populating process is not continuous.  Discrete measurements are made, so that sparsely occupied cells in a histogram will usually have either zero, one, or two points in them.  This means that when computing the partition function $\Gamma$, instead of having a collection of terms that have small $p$ values that are not fully converged, but still change smoothly, there will be a collection of cells that have only one point in them.  If the number of singly occupied cells is denoted by $N_{\{n=1\}}=N_1$, then once $N_1 \approx N_2 2^q \Rightarrow q_c \approx \log (N_1/N_2) / \log 2$, the $n=1$ terms will dominate the sum.  

Notice that if we are computing $\tau$ for $q$ more negative than $q_c$, we will effectively have $\sum n^q \approx N_1 1^q = N_1$.  This quantity will depend on $\ell$, but it is completely independent of $q$.  When finding our slope of the $\sum p^q$ vs. $\log\ell$ curves, we will get $\tau \approx \tau_0$, giving $D_q = \tau/(q-1) = \tau_0/(q-1)$.  If if so happens that $N_1$ varies in just the right manner so that $\tau_0 = D_{-\infty}$, then this will produce the correct $D_q$ spectrum, but of course the sparse areas of the attractor were the areas controlling $N_1$, and it is exceedingly rare for these areas to have converged well enough to produce the correct value of $D_{-\infty}$ at a stage where many cells are still only singly occupied.  The more general behavior is that there will be a small number of mostly isolated singly occupied cells, so that as the histogram is re-binned, $N_1$ will stay mostly constant, then make abrupt drops as singly occupied cells get binned into the same cell as each other at some particular length scale.  This means that often $\tau_0$ is not negative enough, since $N_1$ is staying more constant than it should.  This leads to an underestimate of $D_{-\infty}$, which can produce $D_q$ curves that have $D_{-\infty}$ lower than $D_\infty$, which violates the requirement that all $D_q$ curves decrease monotonically with $q$. %ref?

% If we are computing $D_q$ for a $q$ value past the $q_c$ threshold, then the partition function becomes basically $\Gamma = \sum \frac{p^q}{\ell^\tau} = \frac{N_1(1/N)^q}{\ell^\tau}$.  Note that the factor of $1/N$ that scales between probability and number of cells does not depend on the box size.  This means that when solving $\Gamma = A$ using a log-log plot, that factor will provide a $q$ dependent offset.  Since $\tau$ is the slope of the $\log\sum p^q$ vs. $\log\ell$ curve, this offset will not affect the calculation of $\tau$, and thus will not change the resulting $D_q$.  There will be a different value of the offset for each different value of $q$ selected, but the fitting is performed at fixed $q$.  Effectively, we can solve $\Gamma = B(q, N)$ instead of $\Gamma = A(N)$.  The result is the same.  %Our algorithm described in section \ref{multi} will find a slope, but we can simply examine this new approximate equation and find an actual solution. We have
% % Recall that our algorithm from section \ref{multi} was finding the slope of 
% 
% This section is not finished.  There is a description in the research notes of how exactly $D_q$ becomes independent of $q$, but I have a sneaking suspicion that that result was from before our least-squares fitting, back when we just took a ratio.  It may need to be re-examined.  The basic result, though, is that past the $q_c$ boundary, we have $D_q \approx a\frac{q}{q-1}$, or something similar.
% 
% 
% 
% Consider the sum performed for $\gamma(q,\ell) = \sum p_i^q$.  Of course, we actually sum up $\sum n_i^q$, where $n_i$ is the number of points in the $i^{th}$ cell, but this is completely equivalent, since the calculation of $\tau$ is done by finding the slope of $\log\gamma$ vs $\log\ell$.  The overall factor of $N$ needed to convert between $n$ and $p$ is just a $q$ dependent offset to the graphs.  Now, examine the limit as $q\to -\infty$.  The sum will be dominated by any cells that have $n_i =1$.  Any cells with $n_i > 1$ will make a vanishing contribution to the sum.  Thus, for highly negative $q$, we expect $\gamma(\ell) \approx N_{\{n=1\}}(\ell)$.  The explicit $\ell$ dependence is hard to pin down, but $N_{\{n=1\}}$ does depend on $\ell$, so that we will get some value for the slope of our $\gamma\ell$ plots.  However, notice that this quantity becomes independent of $q$.  The $q$ independence is also hard to explicitly quantify, but empirically, it tends to take over quite quickly, near $q_c\approx -2$.  To the left of there, $\gamma$ (and thus $\tau$) stays more or less fixed at a constant value, but recall that $D_q = \tau/(q-1)$, so that as $q \to -\infty$, $D_q$ now falls off more or less as $1/q$.  This leads to the $D_q$ curves asymptotically approaching some constant value, which is not the correct $D_{-\infty}$.% smooth ``backwards" behavior.  
% 
% Notice that removing the $n=1$ cells does not fix the problem; the $n=2$ cells then play a similar role.  The real problem here is that we have $p$ not smoothly varying in any moderately realistic experiment.  The difference between the least occupied cells and the next least occupied cells is too drastic, and the populations of these two types of cell are too large.  Basically, we need more points per cell, although requiring a higher average occupancy is not quite enough.  We need to have even the sparsest parts of the attractor well-populated enough that there is smooth variation between cells.  This means that very large amounts of data are required, since the sparse areas of the attractor are visited with the least frequency.

% Of course, we started looking for solutions before making sure that we had accurately diagnosed the problem, but I eventually went back to the 1D case and tried using way too little data.  The results are in figure \ref{2010_09_sparse1d}.  The 1D results supported out theory, and suggested that perhaps if I used 1000 points per cell, the curves might switch over to the expected behavior.  This turned out not to be the case.

A phenominological demonstration of this behavior is shown in figure \ref{2010_09_sparse1d}.  Various $D_q$ curves were computed for the logistic map, with different numbers of points used to populate the histogram.  When the histogram is too sparse, the curves asymptotically approach a wide spread of values for $q\to -\infty$.  These values are not related to the actual $D_{-\infty}$ result.

\begin{figure}[!ht]
\includegraphics[width=.9\textwidth]{20100922_progression.png}
\caption[Plots demonstrating the effect of sparse data on $D_q$ convergence.]{Plots of the $D_q$ curve for the logistic map as the total number of points is varied.  There were 10,000 bins in the histogram.  Notice that $D_q$ settles down to the expected curve in the general vacinity of 100 points per cell, although in reality this number is slightly higher, since some cells are empty.  Also notice that although $N_{\{n=1\}}=0$ appears to be a necessary requirement, it is not sufficient.  All these curves were calculated with the naive algorithm. \label{2010_09_sparse1d}}
\end{figure}

This type of error is known as a clipping error, %ref?
and there are several existing methods that attempt to provide better estimates of $D_q$ for negative $q$.  We are most concerned with the Extended Box Algorithm (EBA).   %this file has been included because it contains a figure.

\chapter{New Methods}

This is the second chapter, and here is a fact which we support with a citation\cite{coxlittleoshea}.  Many people put the chapter declaration inside the inputted file\cite{Yorke1989}, in order to be able to reorder things very easily\cite{robot_homotopy}.  

\section{A new definition of dimension?}

Will you unite the disparate definitions of the current science? \lipsum[1]

\subsection{A subsection}

% This is still experimental.  I hear that perhaps the full caption should go above the table, in which case we would do:
% \begin{table}[htp]
% \caption[short version of caption]{Longer, more detailed caption, perhaps containing a real description of the table's contents. \label{table:faketable}}
% \begin{tabular}...
\begin{table}[htp]
\caption{Test Table \label{table:faketable}}
\begin{tabular}{|c|c|c|}
\hline
Item 1 & Item 2 & Item 3\\
\hline
1 & 2 & 3 \\
4 & 5 & 6 \\
\hline
\end{tabular}
\begin{minipage}{0.9\textwidth} ~\\ \footnotesize
Here is a table with nothing in it.  This is the caption for the table.  It's only here to make sure that the list of tables is working correctly, and as just a double check on basic table sort of stuff.
\end{minipage}%clunky.  Need to make this look better.
\end{table}

This text here is the ``actual" location of table \ref{table:faketable}.  Since tables are floating objects, the table itself may have moved around so as to better fit on the page.  I used the options \verb-[htp]-, which should attempt to place the table \verb-h-ere, or if that fails, at the \verb-t-op of this or the next page\footnote{I'm not sure if the top option will still do next page now that we are using the single side option for amsbook.}, or if that fails, on a \verb-p-age containing only floats.  The float page gets purged at the end of each section, I think.  You can also use global options to specify that all floats end up in special places, like pages containing only floats.

\subsection{Another Subsection}

Automatically generated nonsense text: \lipsum[3-5]

\begin{sidewayspage}
\begin{table}[h]
\caption{Sideways Table \label{table:sidetable}}
\begin{tabular}{|c|c|c|ccc|}
\hline
Item 1 & Item 2 & Item 3 & more items & more items & more items\\
\hline
\hline
1 & 2 & 3 & x & x & x\\
4 & 5 & 6 & x & x & y\\
\hline
\end{tabular}
\end{table}
\end{sidewayspage}

Here is some text inserted after the sideways table.  Hopefully the table will float, not forcing a break before this text.  If these lines are right at the top of a new page, something might be wrong. \lipsum[1-3]

\backmatter
\bibliographystyle{plain} % I still need to check to see if this is what was fucking with my formatting.
\bibliography{samplebibfile} %note that this is a separate file
% \bibliography{leifbib} %note that this is a separate file

\appendix

% \input{appendix1.tex}
\chapter{Automatically Generated Supplementary Material} %this will be displayed as an appendix, not a chapter.  Appendices are at the same level in the hierarchy as chapters.

\section{Some Sample Material}

Appendix is a strange name.  Did the name for the written material come before the name of the organ?

\lipsum[5-7]

\section{On Why I Should Work Harder}

Let me address that issue with the following classical arguments: \lipsum[1-2]

\chapter{Another Supplement}

\lipsum[1-3]

\end{document}